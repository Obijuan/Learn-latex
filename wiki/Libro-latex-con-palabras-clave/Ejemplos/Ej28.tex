\documentclass{article}

% Preámbulo estándard
\usepackage[T1]{fontenc}
\usepackage[utf8]{inputenc}
\usepackage[spanish]{babel}

\hyphenation{ma-te-má-ti-cos}

\begin{document}

Como bien afirmó \textit{Galileo}, las \textbf{matemáticas} constituyen un
\textsc{lenguaje universal}. \textsf{Podríamos decir que no sólo constituyen
la \textit{base de todo conocimiento}, sino también de \textbf{cualquier tipo}
de desarrollo científico y tecnológico}. \texttt{De hecho, ciencias como la
\textit{filosofía} o la \textbf{psicología} se valen de 
\textsc{modelos ma\-temáticos} para la resolución de sus problemas} 

\end{document}
