\documentclass{article}

% Preámbulo estándard
\usepackage[T1]{fontenc}
\usepackage[utf8]{inputenc}
\usepackage[spanish]{babel}

% Paquete para el surayado y tachado
\usepackage{soul}


\begin{document}

 Como bien afirmó Galileo, las matemáticas constituyen un lenguaje universal.
 Podríamos decir que no sólo constituyen \underline{la base de todo
 conocimiento}, sino también de cualquier tipo de desarrollo científico y
 tecnológico.
 \textst{Sin embargo} De hecho, ciencias como la filosofía o la 
 psicología se valen de modelos matemáticos para la resolución de sus
problemas.

\end{document}
