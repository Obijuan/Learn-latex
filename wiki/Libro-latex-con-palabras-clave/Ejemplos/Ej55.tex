\documentclass[x11names,table]{article}

% Preámbulo estándard
\usepackage[T1]{fontenc}
\usepackage[utf8]{inputenc}
\usepackage[spanish]{babel}

\usepackage[usenames, dvipsnames]{xcolor}

\begin{document}

% -------- Tabla

%-- Color de las filas
%--    Comienzo-Color filas pares-Color filas impares
\rowcolors{1}{blue!2}{green!20}
\begin{tabular}{ccc}
  %-- lcr: alineacion de cada columna
  %--  l: left, c: center, r: right
  %\hline
  %-- Color de la primera fila
  \rowcolor{red!17} {\bf Item 1} & {\bf Item 2} & {\bf Item 3} \\ \hline
  Elemento 11 & Elemento 12 & Elemento 13 \\ \hline
  Elemento 21 & Elemento 22 & Elemento 23 \\ \hline
  Elemento 31 & Elemento 32 & Elemento 33 \\ \hline
  Elemento 41 & Elemento 42 & Elemento 43 \\ \hline
  Elemento 51 &\cellcolor{black!15} Elemento 52 & Elemento 53 \\ \hline
\end{tabular}

\end{document}
